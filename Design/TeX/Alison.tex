\documentclass[12pt,a4paper]{article}

\usepackage[utf8]{inputenc}
\usepackage[english]{babel}
\usepackage{amsmath}
\usepackage{amsfonts}
\usepackage{amssymb}
\usepackage{graphicx}
\usepackage{float}
\usepackage{tabularx}
\usepackage{xspace}
\usepackage{dirtree}
\usepackage{url}
\usepackage{color}
\usepackage{xcolor}
\usepackage[toc,page]{appendix}
\usepackage[colorlinks=false, pdfborder={0 0 0}]{hyperref}
\usepackage[left=2.30cm, right=1.50cm, top=2.00cm, bottom=1.80cm]{geometry}
\usepackage[scaled]{uarial}
\renewcommand*\familydefault{\sfdefault}
\usepackage[T1]{fontenc}


% USAGE DIRTREE
%==============
%\dirtree{%
%	.1 Root.
%	.2 Level 2 \DTcomment{Level 2 comment}.
%	.3 Level 3 \DTcomment{Level 3 comment}.
%}

% USAGE TABULARX
%================
%\begin{tabularx}{0.95\textwidth}{|l|X|}
%	\hline
%	Header 1	& Header 2		\\
%	\hline
%	Line 1 Left	& Line 1 Right 	\\
%	Line 2 Left	& Line 2 Right 	\\
%	\hline
%\end{tabularx}

%USAGE ITEMIZE WITH ITEMSEP
%==========================
%\begin{itemize}\itemsep-4pt
%\item First Item
%\item Second Item
%\item ...
%\item Last Item
%\end{itemize}

% USAGE APPENDICES
%=================
%\appendices
%\section{Important Equations}

% MATRICES
%=========
% matrix		no borders
% pmatrix		()
% bmatrix		[]
% vmatrix		||
% Vmatrix		|| ||
% BMatrix		{}

% ALIGNMENT
% =========
% \begin{equation}\label{key}
% 	\begin{aligned}
% 		a   &= x_{ij} \\
% 		b_j &= y_j 
% 	\end{aligned}
% \end{equation}

% AUTO COUNTER (the prefix here being "Requirement")
%===================================================
% \newcounter{ReqCounter}
% \newcommand{\req}{\protect\stepcounter{ReqCounter}\textbf{Requirement} \textbf{\theReqCounter.}}

\author{Stanislav Koncebovski}
\title{Project Alison}
\date{2024-10-08}

\begin{document}
	\maketitle
	
	\section{Motivation}
	Working on a genealogy software project we have found out that it would be highly desirable to support different string manipulation routines, such as Soundex, Daitch-Mokotoff Soundex, and a number of others, as well as a number of usual string metrics, such as Levenshtein and similar. \\
	For Python, there exists a brilliant framework called \textbf{abydos} \cite{chrislitabydos}; for C\#, we failed to find a similar library, therefore we have decided to develop it ourselves, at least, partially.
	
	\section{Abydos functionality}
	Below listed are the main routines supported by the \textbf{abydos} framework. Routines that we mean to implement first are rendered red.
	
	\subsection{Phonetic algorithms}
	\begin{itemize}\itemsep-4pt
		\item {\color{red} Robert C. Russell's Index} (\cite{Russell:1917})
		\item {\color{red} American Soundex} (\cite{US:2007}, \cite{Knuth:1998})
		\item Refined Soundex
		\item {\color{red} Daitch-Mokotoff Soundex} (\cite{Mokotoff:1997})
		\item {\color{red} Kölner Phonetik} (\cite{Postel:1969})
		\item NYSIIS
		\item Match Rating Algorithm
		\item {\color{red} Metaphone} (\cite{Philips:1990}, \cite{Philips:1990b}, \cite{Kuhn:1995})
		\item {\color{red} Double Metaphone} (\cite{Philips:2000})
		\item Caverphone
		\item Alpha Search Inquiry System
		\item {\color{red} Fuzzy Soundex} \cite{Holmes:2002}
		\item Phonex
		\item Phonem
		\item {\color{red} Phonix} (\cite{Pfeifer:2000}, \cite{Christen:2011}, \cite{Kollar:2007})
		\item SfinxBis
		\item phonet
		\item Standardized Phonetic Frequency Code
		\item Statistics Canada
		\item Lein
		\item Roger Root
		\item Oxford Name Compression Algorithm (ONCA)
		\item Eudex phonetic hash
		\item Haase Phonetik
		\item Reth-Schek Phonetik
		\item FONEM
		\item Parmar-Kumbharana
		\item Davidson's Consonant Code
		\item SoundD
		\item PSHP Soundex/Viewex Coding
		\item an early version of Henry Code
		\item Norphone
		\item Dolby Code
		\item Phonetic Spanish
		\item Spanish Metaphone
		\item MetaSoundex
		\item SoundexBR
		\item NRL English-to-phoneme
		\item Beider-Morse Phonetic Matching
	\end{itemize}
	
	\subsubsection{String distance metrics}
	\begin{itemize}\itemsep-4pt
		\item String distance metrics
		\item {\color{red} Levenshtein distance}  (\cite{Levenshtein:1965}, \cite{Levenshtein:1966})
		\item Optimal String Alignment distance
		\item {\color{red} Levenshtein-Damerau distance} (\cite{Damerau:1964})
		\item Hamming distance
		\item Tversky index
		\item Sørensen–Dice coefficient \& distance
		\item {\color{red} Jaccard similarity coefficient \& distance} (\cite{Jaccard:1901}, \cite{Tanimoto:1958}, \cite{Tversky:1977})
		\item overlap similarity \& distance
		\item Tanimoto coefficient \& distance
		\item Minkowski distance \& similarity
		\item Manhattan distance \& similarity
		\item Euclidean distance \& similarity
		\item Chebyshev distance
		\item {\color{red} cosine similarity \& distance} (\cite{Otsuka:1936}, \cite{Ochiai:1957})
		\item {\color{red} Jaro distance} (\cite{Jaro:1989}, \cite{Winkler:1990})
		\item {\color{red} Jaro-Winkler distance (incl. the strcmp95 algorithm variant)}
		\item Longest common substring
		\item Ratcliff-Obershelp similarity \& distance
		\item Match Rating Algorithm similarity
		\item Normalized Compression Distance (NCD) \& similarity
		\item Monge-Elkan similarity \& distance
		\item Matrix similarity
		\item Needleman-Wunsch score
		\item Smith-Waterman score
		\item Gotoh score
		\item Length similarity
		\item Prefix, Suffix, and Identity similarity \& distance
		\item Modified Language-Independent Product Name Search (MLIPNS) similarity \& distance
		\item Bag distance
		\item Editex distance
		\item Eudex distances
		\item Sift4 distance
		\item Baystat distance \& similarity
		\item Typo distance
		\item Indel distance
		\item Synoname
	\end{itemize}	

	
	\bibliographystyle{ieeetr}
	\bibliography{Alison}
\end{document}	